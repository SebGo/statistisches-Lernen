\documentclass[10pt,a4paper]{article}
\usepackage{ngerman}
\usepackage[utf8]{inputenc}
\usepackage{amsmath}
\usepackage{amsfonts}
\usepackage{amssymb}
\usepackage{graphicx}
\usepackage{geometry}
\usepackage{tikz}
\usepackage{todonotes}
\usepackage{amsthm}
\usepackage{array}
\usepackage{ wasysym }

%thick lines in tables
\makeatletter
\newcommand{\thickhline}{%
\noalign{\ifnum0=`}\fi \hrule height 1pt
\futurelet\reserved@a\@xhline
}
\newcolumntype{"}{@{\hskip\tabcolsep\vrule width 1pt\hskip\tabcolsep}}
\makeatother

\geometry{a4paper, top=25mm, left=40mm, right=25mm, bottom=30mm,
headsep=10mm, footskip=12mm}
\title{Übung 5 - Statistisches Lernen\\ \large{Wintersemester 17/18}}
\date{}

\begin{document}
\maketitle
\section*{Aufgabe 1}

\begin{tabbing}
	\hspace{90pt}\=\hspace{30pt}\=\hspace{130pt}\=\kill
	\(p_{nn}\)\> \(\leq / >\)   \> \(\tilde{\alpha}_{m-j+1}\)  \> Aussage\\	
	\(p_{1} = 0,001 = p_{6}\)\> \(\leq\) \> \(\tilde{\alpha} _{m-1+1} = \frac{\alpha}{10} = 0,005\)  \> \(H_0\) ablehnen \\	
	\(p_{2} = 0,003 = p_{1}\)\> \(\leq\)  \>\(\tilde{\alpha} _{m-2+1} = \frac{\alpha}{9} = 0,005\)  \>  \(H_0\) ablehnen\\	
	\(p_{3} = 0,004 = p_{9}\)\> \(\leq\)  \> \(\tilde{\alpha} _{m-3+1} = \frac{\alpha}{8} = 0,00625\) \> \(H_0\) ablehnen \\	
	\(p_{4} = 0,005 = p_{2}\)\> \(\leq\)  \> \(\tilde{\alpha} _{m-4+1} = \frac{\alpha}{7} = 0,007\) \> \(H_0\) ablehnen \\	
	\(j^* \text{ } p_{5} = 0,008 = p_{4}\)\>  \(\leq\) \> \(\tilde{\alpha} _{m-5+1} = \frac{\alpha}{6} = 0,008\bar{3}\)  \> \(H_0\) ablehnen\\	
	\(p_{6} = 0,011 = p_{7}\)\> \(>\)  \> \(\tilde{\alpha} _{m-6+1} = \frac{\alpha}{5} = 0,01\) \> \(H_0\) \underline{nicht} ablehnen\\	
	\(p_{7} = 0,036 = p_{5}\)\> \(>\) \>\(\tilde{\alpha}_{m-7+1} = \frac{\alpha}{4} = 0,0125\)  \>  \(H_0\) \underline{nicht} ablehnen\\	
	\(p_{8} = 0,040 = p_{10}\)\>\(>\)  \>\(\tilde{\alpha} _{m-8+1} = \frac{\alpha}{3} = 0,01\bar{6}\)  \> \(H_0\) \underline{nicht} ablehnen \\	
	\(p_{9} = 0,049 = p_{3}\)\> \(>\) \>  \(\tilde{\alpha} _{m-9+1} = \frac{\alpha}{2} = 0,025\)\> \(H_0\) \underline{nicht} ablehnen \\	
	\(p_{10} = 0,059 = p_{}\)\> \(>\) \>  \(\tilde{\alpha} _{m-10+1} = \frac{\alpha}{1} = 0,05\)\>  \(H_0\) \underline{nicht} ablehnen\\		
\end{tabbing} 
\[j^* = \max\{ \max\{ j \in\{1,\ldots,m\} : p(k)\leq \tilde{\alpha}_{m-j+1} \text{für alle} k \in\{1, \ldots, j\} \}, 0 \} \]


\section*{Aufgabe 2}
\begin{tabbing}
	\hspace{90pt}\=\hspace{30pt}\=\hspace{130pt}\=\kill
	\(p_{nn}\)\> \(\leq / >\)   \> \(\title{\alpha}_{j}\)  \> Aussage\\	
	\(p_{1} = 0,001 = p_{6}\)	\> \(\leq\)	\> \(j\frac{\alpha}{m} = 1 *\frac{0.05}{10} = 0,005\)  \> \(H_0\) ablehnen \\	
	\(p_{2} = 0,003 = p_{1}\)	\> \(\leq\)	\> \(2 \frac{0,05}{9} = 0,01\)  \>  \(H_0\) ablehnen\\	
	\(p_{3} = 0,004 = p_{9}\)	\> \(\leq\)	\> 0,015	\> \(H_0\) ablehnen \\	
	\(p_{4} = 0,005 = p_{2}\)	\> \(\leq\)	\> 0,020	\> \(H_0\) ablehnen \\	
	\(p_{5} = 0,008 = p_{4}\)	\> \(\leq\)	\> 0,025	\> \(H_0\) ablehnen\\	
	\(p_{6} = 0,011 = p_{7}\)	\> \(\leq\) \> 0,030	\> \(H_0\) ablehnen\\	
	\(p_{7} = 0,036 = p_{5}\)	\> \(\leq\) \> 0,035	\>  \(H_0\) ablehnen\\	
	\(p_{8} = 0,040 = p_{10}\)	\>\(\geq\)  \> 0,040	\>\(H_0\) ablehnen\\	
	\(p_{9} = 0,049 = p_{3}\)	\> \(>\) 	\>  0,045 \(j^* = 9\)	\> \(H_0\) \underline{nicht} ablehnen \\	
	\(p_{10} = 0,059 = p_{}\)	\> \(>\) 	\>  0,050	\>  \(H_0\) \underline{nicht} ablehnen\\		
\end{tabbing} 	
\[j^* = \min\{ \min\{ j \in \{1, \ldots, m\} : p(k) > \tilde{\alpha}_k \text{für alle } k \in \{j, \ldots,m\} \}, m+1\}\]


\section*{Aufgabe 3}
\begin{itemize}
	\item \( m \in \mathbb{N} \setminus \{1\} \)
	\item \(\alpha \in (0,1)\)
\end{itemize}
zu zeigen:
\[
	\frac{\alpha}{m} < 1-(1-\alpha)^{\frac{1}{m}} \text{ "`Taylor - Entwicklung"'}
\]
Beweis:
\begin{align*}
	f(\alpha) &= 1-(1-\alpha)^{\frac{1}{m}} &\text{für } m \in \mathbb{N} \setminus \{1\}\\
	\text{Taylor: } f(\alpha) &= f(0) + f^\prime(0) \alpha + 	
	\underbrace{\dfrac{f^\prime(\xi) (\alpha - \xi)^2}{2} \alpha}_{\text{Restglied}}	
	&\text{für ein } \xi \in (0,\alpha)\\
	\rightarrow 1-(1-\alpha)^\frac{1}{m}  &= f(\alpha) \\
	&=_{\text{Taylor}} 0 + 
	\underbrace{\frac{1}{m}(1-\alpha)^{\frac{1}{m}-1} |_{\alpha=0}}_{f\prime(0)}
	\alpha - \frac{1}{2}
	\frac{1}{m} (\frac{1}{m} -1) (1- \alpha)^{\frac{1}{m} - 2} |_{\alpha= \xi} (\alpha - \xi)^2 \alpha \\
	&=\frac{1}{m}\alpha \underbrace{- 
	\underbrace{
\underbrace{\frac{1}{2}}_{>0}
\underbrace{\frac{1}{m}}_{>0}
\underbrace{(\frac{1}{m} -1)}_{<0}
\underbrace{(1-\xi)^{\frac{1}{m} - 2}}_{>0}
\underbrace{\alpha-\xi)^2}_{>0}
\underbrace{\alpha}_{>0}}_{<0}}_{>0}\\
	&> \frac{\alpha}{m}
\end{align*}



\end{document}