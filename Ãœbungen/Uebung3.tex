\documentclass[10pt,a4paper]{article}
\usepackage{ngerman}
\usepackage[utf8]{inputenc}
\usepackage{amsmath}
\usepackage{amsfonts}
\usepackage{amssymb}
\usepackage{graphicx}
\usepackage{geometry}
\usepackage{tikz}
\usepackage{todonotes}
\usepackage{amsthm}
\usepackage{array}

%thick lines in tables
\makeatletter
\newcommand{\thickhline}{%
    \noalign {\ifnum 0=`}\fi \hrule height 1pt
    \futurelet \reserved@a \@xhline
}
\newcolumntype{"}{@{\hskip\tabcolsep\vrule width 1pt\hskip\tabcolsep}}
\makeatother

\geometry{a4paper, top=25mm, left=40mm, right=25mm, bottom=30mm,
	headsep=10mm, footskip=12mm}
\title{Uebung 3 - Statistisches Lernen\\ \large{Wintersemester 17/18}}
\date{}

\begin{document}
	\maketitle
\section{Aufgabe 1}
\subsection{Aufgabe 1a}
\[ T = \frac{\bar{x_1} - \bar{x_2}}{s \sqrt{\frac{1}{n_1} + \frac{1}{n_2}}} \hspace{0.5cm},\hspace{0.5cm} s^2 = \frac{(n_1 - 1) s_1^2 + (n_2 -1) s_2^2}{n_1 + n_2 -2} \]
\[ T = \frac{28.7 - 26.2}{\sqrt{\frac{2.6^2 + 3.1^2}{170}}} \approx \frac{2.5}{0.31} \approx 8.06 \]
\[ \Rightarrow p= 1.4 \cdot 10^{-14} \]
$(Freiheitsgrade: n_1 + n_2 -2 = 338)$ \\

\subsection{Aufgabe 1b}
(Parität = Anzahl von Geburten) 
\[ T = \frac{\bar{x}_1 - \bar{x}_2 }{\sqrt{\frac{s_1^2}{n_1} + \frac{s_2^2}{n_1}}} = \frac{0.1}{\sqrt{\frac{0.8^2 + 0.9^2}{170}}} \approx 1.08 \]

\subsection{Aufgabe 1c}
Mit Standardnormalverteilung $\rightarrow p=0.28$ (relative Differenz ist $10^{-3}$)\todo{Die t-Verteilung nähert sich mit steigenden FG der Standardnormalverteilung an}

\subsection{Aufgabe 1d}
Bei randomisierter Studie stimmt die Nullhypothese per Definition (da die Zuordnung zu einer Gruppe zufällig ist). Wir können nicht $H_0$ verwerfen!

\section{Aufgabe 2}
\subsection{Aufgabe 2a}
OR early
\[ \left(\frac{7}{163} \middle/ \frac{15}{155} \right)^{-1} = 2.25  \] \todo{Ohne $^{-1}$ geht auch, dann kriegt man den Kehrwert raus}
\vspace{1cm}
\[KI: OR \cdot e^{\pm z_{a/2} \cdot \sqrt{\frac{1}{n_{11}} + \frac{1}{n_{21}} + \frac{1}{n_{12}} + \frac{1}{n_{22}}}} = [0.89,5.68] \]
$(Z_{0.025}=1.96)$
\[ [0.89,5.68] \rightarrow p > 0.05 \] 
OR late [95\% KI] $= 2.07 \ [0.37,11.2] \rightarrow p > 0.05 $

\subsection{Aufgabe 2b}
Beobachtet (early)
\begin{tabular}{c|c"c}
 7 & 15 & 22 \\ \hline
 163 & 155 & 318 \\ \thickhline
 170 & 170 & 340 
\end{tabular} 
\newline
Erwartet 
\begin{tabular}{c|c"c}
 11 & 11 & 22 \\ \hline
 159 & 159 & 318 \\ \thickhline
 170 & 170 & 340 
\end{tabular}
\newline
$e_{ij}=\frac{n_{i\cdot} n_{\cdot j}}{n_{\cdot \cdot}}$
\[ \chi^2 = 2 \cdot \left( \frac{4^2}{11} \right) + 2 \cdot \left( \frac{4^2}{159} \right) = 3.11 \]
$\Rightarrow p=0.078 $
\subsection{Aufgabe 2c}
\paragraph{Nebenbemerkung Symmetrie}
\[ p = \frac{\binom{n_{1\cdot}}{n_{11}} \binom{n_{2\cdot}}{n_{22}}}{ \binom{n_{\cdot \cdot}}{n_{\cdot1}}} 
\stackrel{?}{=} \frac{\binom{n_{\cdot1}}{n_{11}} \binom{n_{\cdot2}}{n_{22}}}{ \binom{n_{\cdot \cdot}}{n_{1\cdot}}} \]
\[ \binom{n_{\cdot \cdot}}{n_{\cdot1}} \stackrel{?}{=} \binom{n_{\cdot \cdot}}{n_{\cdot2}} \]
\[ \binom{n_{\cdot \cdot}}{n_{\cdot1}} = \frac{(n_{11} + n_{12} + n_{21} + n_{22})!}{(n_{11} + n_{21})!(n_{12} + n_{22})!} = \binom{n_{\cdot \cdot}}{n_{\cdot2}} \]
\[ p = \frac{(n_{11} + n_{12})!(n_{21} + n_{22})!(n_{11} + n_{21})!(n_{12} + n_{22})!}{n_{11}! n_{12}! n_{21}! n_{22}! (n_{11} + n_{12} + n_{21} + n_{22})!}
= \frac{\binom{n_{\cdot1}}{n_{11}} \binom{n_{\cdot2}}{n_{22}}}{ \binom{n_{\cdot \cdot}}{n_{1\cdot}}} \]
Man kann also tauschen was Reihe und Spalte ist (Ende Nebenbemerkung) \\

Für 
\begin{tabular}{c|c"c}
 2 &  4 & 6 \\ \hline
 168 & 166 &  334 \\ \thickhline
 170 & 170 & 340
\end{tabular}
\vspace{0.2cm}

\[ p_1 = \frac{\binom{6}{2}\binom{334}{166}}{\binom{340}{170}} = \frac{\frac{6!}{2!4!} \frac{334!}{166!168!}}{\frac{340!}{170!170!}} 
= \frac{15 \cdot \frac{1}{1}}{\frac{340 \cdot 339 \cdot \dotsc \cdot 335}{170 \cdot 169 \cdot 168 \cdot 167 \cdot 170 \cdot 169}}
= 0.235 \]

Für 
\begin{tabular}{c|c"c}
 1 &  5 & 6 \\ \hline
 169 & 165 &  334 \\ \thickhline
 170 & 170 & 340
\end{tabular}
\vspace{0.2cm}
\[ p_2 = \frac{\binom{6}{1}\binom{334}{169}}{\binom{340}{170}} = 0.0923 \]

Für 
\begin{tabular}{c|c"c}
 0 &  6 & 6 \\ \hline
 170 & 164 &  334 \\ \thickhline
 170 & 170 & 340
\end{tabular}
\vspace{0.2cm}
\[ p_3 =  0.0169 \]

\[p = 2 \cdot (p_1 + p_2 + p_3) = 0.685 \]

\subsection{Aufgabe 2 d}
OR[95\% KI] : 2.25 [0.98,5.13]\\
aus dem $\chi^2$-Test $\rightarrow$ p=0.0485
\end{document}

