\documentclass[10pt,a4paper]{article}
\usepackage[latin1]{inputenc}
\usepackage{amsmath}
\usepackage{amsfonts}
\usepackage{amssymb}
\usepackage{graphicx}
\usepackage{geometry}
\usepackage{tikz}
\usepackage{todonotes}
\usepackage{amsthm}

\geometry{a4paper, top=25mm, left=40mm, right=25mm, bottom=30mm,
	headsep=10mm, footskip=12mm}
\title{�bung 1 - Statistisches Lernen\\ \large{Wintersemester 17/18}}
\date{}

\begin{document}
	\maketitle
	\section*{Aufgabe 1}
	Vereinfachen Sie folgende Ausdr�cke
	\begin{itemize}
		\item $(A+B)(B+C) = B + AC $
		\item$  (A+B)(A+\bar{B}) = A + B \bar{B} = A $
		\item $(A+B)(\bar{A}+B)(A+\bar{B}) = B+(A \bar{A}) (A + \bar{B}) = B  (A + \bar{B}) = (BA) + (B \bar{B}) = BA $
	\end{itemize}
	
	\section*{Aufgabe 2}
	Eine faire M�nze wird 7 Mal geworfen
	\begin{itemize}
		\item[(a)] Wie gro� ist die Wahrscheinlichkeit, das Ergebnis ZWZZWWZ zu erhalten
	\end{itemize}
	\[P(ZWZZWWZ) = 0.5 ^7 = \frac{1}{128}\] oder
	\[P(ZWZZWWZ) = \frac{n}{m} \text{ mit } m = V_7^2 = 2^7 = 128 \text{ und } n=1 \]
	
	\begin{itemize}
		\item[(b)] Wie gro� ist die Wahrscheinlichkeit, genau 4 mal Z zu erhalten
	\end{itemize}
 	\[ P(Z = 4) = \binom{7}{4} 0.5^4 (1-0.5)^3 = \frac{35}{128}\]
 	oder 
 	\[ P(Z = 4) = \frac{n}{m} = \frac{35}{128} \text{ mit } n=P_7^{4,3} = \frac{7!}{4! 3!} = 35 \text{ und } n=128\]
 	
 	\begin{itemize}
 		\item[(c)] Wie gro� ist die Wahrscheinlichkeit, mindestens 4 mal Z zu erhalten
 	\end{itemize}
 	 \begin{align*}
 	P(Z \geq 4) &= P(Z=4) + P(Z=5) +P(Z=6) +P(Z=7) \\
 	&= \binom{7}{4} 0.5^4 (0.5)^3 + \binom{7}{5} 0.5^5 (0.5)^2 +  \binom{7}{6} 0.5^6 0.5 + \binom{7}{7} 0.5^7 = 0.5
 	\end{align*} 
 	oder 
 	\[P(Z \geq 4) = P_7^{4,3} + P_7^{5,2} P_7^{6,1} +P_7^{7} = 0.5\]


\section*{Aufgabe 3}
Leiten Sie die Bayessche Formel her
\begin{align*}
	P(A|B) &= \frac{P(AB)}{P(B)} \text{ mit } \bigcup B_i = \Omega; B_i,B_j = \emptyset \text{ f�r } i \neq j \\\
	P(B_i|A) &= \frac{P(AB_i)}{P(A)} = \frac{P(AB_i)}{P(A \cup_i B_i)} = \frac{P(AB_i)}{P(\cup AB_i)} & AB_iAB_j = \emptyset\\
	&\rightarrow \frac{P(AB_i)}{\sum_{j}P( AB_j)} = \frac{P(A|B_i) P(B_i)}{\sum_{j=1}^{n}P( A|B_j) P(B_j)}
\end{align*}


\section*{Aufgabe 4}
Ab der 21. Schwangerschaftswoche wird unter Umst�nden die ''short-term variation'' als Teil der Kardiotokographie gemessen. Bei bestimmten Werten wird der F�tus als krank angesehen und die Geburt wird eingeleitet\\

Sowohl Sensitivit�t als auch Spezifit�t betragen 98\%. Die Pr�valenz liegt bei Patientinnen mit einer solchen Messung bei 1\%  
\begin{itemize}
	\item[(a)] Was ist der positiv-pr�diktive Wert (engl. PPV)?
\end{itemize}
\begin{align*}
PPV &= \frac{\text{Anzahl der richtig positiven}}{\text{Anzahl der Richtig positiven + Anzahl der falsch positiven}} \\
&= \frac{\text{Sensitivit�t} * \text{Pr�valenz}}{\text{Sensitivit�t}*\text{Pr�valenz} + (1 - \text{Spezifit�t})*(1-\text{Pr�valenz})}\\
P(D^+|T^+) &=\frac{P(T^+|D^+) P(D^+)}{P(T^+|D^+)P(D^+) + P(T^+|D^-) P(D^-)}\\ 
&= \frac{0.98*0.01}{0.98*0.01 + 0.02* 0.99} = 0.331
\end{align*}

\begin{itemize}
	\item[(b)] Wie hoch ist der PPV, wenn die Messun zwei Mal durchgef�hrt wird und zwar unter der Annahme, dass die Messungen unkorreliert sind?
\end{itemize}
\todo{Aufgabe 4b}

\section*{Aufgabe 5}
Berechnen Sie die Varianz f�r die Binomialverteilung!
\[\text{Wahrscheinlichkeitsfunktion: } \binom{n}{k}p^k(1-p)^{n-k}\]

\todo{Aufgabe 5}

\section*{Aufgabe 6}
Berechnen Sie die Varianz f�r die Normalverteilung!
\[\text{Wahrscheinlichkeitsdichte: } \dfrac{1}{\sigma \sqrt{2 \pi}} \exp(- \frac{(x-a)^2}{2\sigma^2})\]
\[ E(X) = \dfrac{1}{\sigma \sqrt{2 \pi}} \exp(- \frac{(x-a)^2}{2\sigma^2}) = a \]
\[ V(X) = E((X - E(X))^2) \]
\[ V(X) = \int_{- \infty}^{\infty} (X-a)^2 e^{-\frac{(x-a)^2}{2 \sigma}} \, \partial x \]
\[ = \frac{1}{\sigma \sqrt{2 \pi}} \int_{-\infty}^{\infty} \sigma^2 y^2 e^{- \frac{y^2}{2}} \sigma \, \partial y \] 
%(e^{-\frac{y^2}{2})'zye^{-\frac{y^2}{2}}\]
Nebenrechnung: \[y = \frac{x-a}{\sigma} \, \frac{\partial x}{\partial y} = \frac{1}{\sigma} \]
\[ \rightarrow dx = \sigma dy \]
Nebenrechnung: \[ (uv)'=uv' + u'v \] \\
\[ \int_{a}^{b}uv'=[uv]_{a}^{b} - \int_{a}^{b}u'v \]

\[ V(X) = \frac{a^2}{\sqrt{2 \pi}} \left( \underbrace{ \left[ y \left( y e^{\frac{y^2}{2}}\right) \right]_{- \infty}^{\infty}}_{=0} - \underbrace{\int_{- \infty}^{\infty} 1 \left( -e^{-\frac{y^2}{2}} \right) dy  }_{=\sqrt{2\pi}} \right) = \sigma^2 \]

\todo{Ich hoffe ich hab das alles richtig abgeschrieben, der Typ hat quer �ber die Tafel tausend Nebenrechnungen hingekritzelt etc.}
\end{document}