\documentclass[10pt,a4paper]{article}
\usepackage[latin1]{inputenc}
\usepackage{amsmath}
\usepackage{amsfonts}
\usepackage{amssymb}
\usepackage{graphicx}
\usepackage{geometry}
\usepackage{tikz}
\usepackage{todonotes}
\usepackage{amsthm}

\geometry{a4paper, top=25mm, left=40mm, right=25mm, bottom=30mm,
	headsep=10mm, footskip=12mm}
\title{Uebung 2 - Statistisches Lernen\\ \large{Wintersemester 17/18}}
\date{}

\begin{document}
	\maketitle
\section*{Aufgabe 1}
siehe �bung1 da dort Aufgabenstellung steht
\section*{Aufgabe 2}
\begin{tabular}{c|cccccccccc|c}
A & 0 & 0 & 0 & 114 & \multicolumn{6}{c|}{$\vdash $ MW = 6.64 $ \dashv$} &170 \\ \hline
B & 0 & 0 & 0 & 108 & \multicolumn{6}{c|}{$\vdash $ MW = 7.88 $ \dashv$} &170 \\ \hline
Anzahl FG & 0 & 1 & 2 & 3 & 4 & 5 & 6 & 7 & 8 & 9..... &  \\ 
\end{tabular} 

Woher kennt man diese Mittelwerte?
\[ \frac{1}{170} \sum\limits_{i=1}^{170} n_i^A \]
\[ = \frac{1}{170} \left( \sum\limits_{i=1}^{114} 3 + \sum\limits_{i=115}^{170} n_i^A \right) \]
\[ \Rightarrow \frac{1}{170} \sum\limits_{i=115}^{170} n_i^A = \frac{1}{56} \left[ 170 \cdot 4.2 - 3 \cdot 114 \right] \]

\section*{Aufgabe 3}
Anzahl der Diabetiker steigt mit wachsendem Steatosegrad \\
Wenn man die Diabetiker nach Krankheiten aufschl�sselt verschwindet dieser Trend

\end{document}