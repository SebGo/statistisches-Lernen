\documentclass[10pt,a4paper]{article}
\usepackage{ngerman}
\usepackage[utf8]{inputenc}
\usepackage{amsmath}
\usepackage{amsfonts}
\usepackage{amssymb}
\usepackage{graphicx}
\usepackage{geometry}
\usepackage{tikz}
\usepackage{todonotes}
\usepackage{amsthm}
\usepackage{array}
\usepackage{ wasysym }

%thick lines in tables
\makeatletter
\newcommand{\thickhline}{%
\noalign{\ifnum0=`}\fi \hrule height 1pt
\futurelet\reserved@a\@xhline
}
\newcolumntype{"}{@{\hskip\tabcolsep\vrule width 1pt\hskip\tabcolsep}}
\makeatother

\geometry{a4paper, top=25mm, left=40mm, right=25mm, bottom=30mm,
headsep=10mm, footskip=12mm}
\title{Uebung 4-Statistisches Lernen\\ \large{Wintersemester 17/18}}
\date{}

\begin{document}
\maketitle
\section*{Aufgabe 1}
Quadratisches Modell:
\[
  Y_i = ax_i^2 + bx_i + c + \epsilon_i \text{   mit   } i\in \{1,\ldots, n\}
\]
Wie könnte man die unbekannten Parameter a, und c schätzen?\\
Kann man die jeweiligen Schätzer explizit angeben\\
\begin{align*}
  Y_i &= h (x_i^{(1)}, \ldots, x_i^{(k)}, \theta_1, \ldots, \theta_p)+ \epsilon_i\\
  Y_i &= ax_i^2 + bx_i + c + \epsilon_i\\
  KQ &:= \sum_{i=1}^{n}{(Y_i - (ax_i^2 + bx_i + c))}^2 \rightarrow_{a,b,c \in \mathbb{R}} \min
\end{align*}

\begin{align*}
  \frac{\partial KQ}{\partial a} &= 2 \sum_{i=1}^{n}(y_i - ax_i^2 - bx_i - c) (-x_i^2) = 0\\
  \frac{\partial KQ}{\partial b} &= 2 \sum_{i=1}^{n}(y_i - ax_i^2 - bx_i - c) (-x_i)= 0\\
  \frac{\partial KQ}{\partial c} &= 2 \sum_{i=1}^{n}(y_i - ax_i^2 - bx_i - c) (-1)= 0
\end{align*}
\begin{align*}
  \rightarrow_{(1)} \sum_{i=1}^{n} x_i^2y_i &=
  a \sum_{i=1}^{n} x_i^4 + b \sum_{i=1}^{n}x_i^3 + c\sum_{i=1}^{n}x_i^2 = 0\\
  \rightarrow_{(2)} \sum_{i=1}^{n} x_{iy} &=
  a \sum_{i=1}^{n} x_i^3 + b \sum_{i=1}^{n}x_i^2 - c + \sum_{i=1}^{n}x_i = 0\\
  \rightarrow_{(3)} \sum_{i=1}^{n}y_i &=
  a \sum_{i=1}^{n} x_i^2 + b \sum_{i=1}^{n}x_i + n  = 0\\
  \ldots\\
  \rightarrow_{(3)} \hat{c} &= \underbrace{\frac{1}{n}\sum_{i=1}^{n} y_i}_{=: \overline{y_n}} - \hat{a} \underbrace{\frac{1}{n} \sum_{i=1}^{n} x_i^2}_{=:\overline{x^2_n}}  - \underbrace{\hat{b} \frac{1}{n} \sum_{i=1}^{n}x_i}_{=:\overline{x_n}}
\end{align*}
\begin{align*}
  \hat{a} &= \frac{
    (\overline{y_n x_n^2} - \overline{y_n} \overline{x_n^2}) \cdot
    (\overline{x_n^2} - {(\overline{x_n})}^2) -
    (\overline{y_n x_n} - \overline{y_n} \overline{x_n}) \cdot
    (\overline{x_n^3} - \overline{x_n} \cdot \overline{x_n^2})
  }{
    (\overline{x_n^4} - {(\overline{x_n^2})}^2)
    (\overline{x_n^2} -{(\overline{x_n})}^2)-
    {(\overline{x_n^3} - \overline{x_n} \cdot \overline{x_n^2})}^2
  }\\
  \hat{b} &= \frac{
    \overline{xy_n} -
    \hat{a}(\overline{x^3_n} -\overline{x^2_n} \overline{x_n} -
    \overline{x_n} \overline{y_n})
  }{ \overline{x^2_n} - \overline{x_n}^2
  }\\
  \hat{c} &= \overline{y_n} - \hat{a} \overline{x^2_n} - \hat{b}\overline{x_n}\\
\end{align*}

\newpage
\section*{Aufgabe 2}
Welche der folgenden funktionalen Zusammenhänge lassen sich linearisieren?

Schätzen der unbekannten Parameter:
KQM
\[
\sum_{i = 1}^{n}(y_i - h (x_i^{(1)}, \ldots, x_i^{(k)}, \theta_1, \ldots, \theta_p)+ \epsilon_i) \rightarrow_{\theta_1,\ldots \theta_p \in \mathbb{R}} \min
\]

Linearisierung
\begin{itemize}
  \item h in linearen Ausdruck transformieren
  \item[\( \rightarrow \)] KQM aus der linearen Regression
  \item[\( \rightarrow \)] Rücktransformation des linearen Modell
  \item[ABER] \( \epsilon_i \) unter Umständen nicht mehr normalverteilt und nicht mehr additiv
  \item [\( \rightarrow \)] Residualanalyse

\end{itemize}
I Reziproke
II loglog\ldots
III exp\ldots \\\\

(a)
\[
y = \frac{\theta_1 x}{\theta_2 + x}\]
\[
\underbrace{\frac{1}{y}}_{:=\tilde{y}}	= \frac{\theta_2 + x}{\theta_1 x}
=\underbrace{ \frac{\theta_2}{\theta_1}}_{:=\tilde{\theta_2}}  \underbrace{\frac{1}{x}}_{:=\tilde{x}} + \underbrace{\frac{1}{\theta_1}}_{:=\tilde{\theta_1}}
\]

(b)

\begin{align*}
	y &= \exp(
	  -\theta_1(x^{(1)}))^
	  {\theta_2 \exp(-\frac{\theta_3}{x^{(2)}})}\\
	\log(y) &= {-\theta_1( x^{(1)} ) }^{\theta_2 \exp(-\frac{\theta_3}{x^{(2)}})}\\
	\log(\log(y)) &=
	\log((-\theta_1) + {(x^{(1)})}^{\theta_2} \exp(- \frac{\theta_3}{x^{(2)}}))\\
	\underbrace{\log(\log(y))}_{\tilde{y}} &=
		\underbrace{\log(-\theta_1)}_
			{\underbrace{:= \tilde{\theta_1}}_
				{(\theta_1>0)}} +
		\underbrace{:= \log((x^{(1)})^{\theta_2})}_
			{\underbrace{\theta_2}_
			 	{:= \tilde{\theta_2}}
			\underbrace{\log(x^{(1)}))}_
				{\underbrace{:= \tilde{x^{(1)}}}_
					{(x^{(1)}>0)}
				}
			}
		\underbrace{-\frac{\theta_3}{x^{(3)}}}_
			{\underbrace{-\theta_3}_
				{:= \tilde{\theta_3}}
			\underbrace{\frac{1}{x^{(2)}}}_
				{:= \underbrace{\tilde{x^{(2)}}}_
					{(x^{(2)} \neq 0)}
				}
		}\\
	\tilde{y} &= \tilde{\theta_1} +  \tilde{\theta_2}  \tilde{x^{(1)}} + \tilde{\theta_3} \tilde{x^{(2)}}
\end{align*}

\newpage
\section*{Aufgabe 3}
\[
  Y_i = \theta_0 + \theta_1 x_i + \theta_2 x_i^2 + \theta_3 x_i^3 + \theta_4 {(x_i - \alpha)}_+^3 +\epsilon_i
  \text{ mit } i \in \{1, \ldots , n\}
\]
Polynom der Ordnung \(M = 3\)
\[
  f(x):= \theta_0 + \theta_1 x_i + \theta_2 x_i^2 + \theta_3 x_i^3 + \theta_4 {(x_i - \alpha)}_+^3 +\epsilon_i
\]
\[
  Z_4=\begin{cases}
    Z, Z \geq 0 \\
   	0  \text{ sonst}
\end{cases}
\]
\( \rightarrow \) bis zu welcher Ableitung stetig??\\
1.Fall \( x< \alpha \)
\begin{align*}
    {f(x)}_{<} &:= \theta_0 + \theta_1 x + \theta_2 x^2 + \theta_3 x^3\\
    {f(x)}_{<}^{\prime} &:= \theta_1 + 2\theta_2 x + 3\theta_3 x^2\\
    {f(x)}_{<}^{\prime \prime} &:= 2\theta_2 + 6 \theta_3 x\\
    {f(x)}_{<}^{\prime \prime \prime} &:= 6 \theta_3\\
\end{align*}
2.Fall \( x \geq \alpha \)
\begin{align*}
    {f(x)}_{\geq} &:= \theta_0 + \theta_1 x + \theta_2 x^2 + \theta_3 x^3 + \theta_4 {(x - \alpha)}^3\\
    {f(x)}_{\geq}^{\prime} &:= \theta_1 + 2\theta_2 x + 3\theta_3 x^2 + 3 \theta_4 {(x-\alpha)}^2\\
    {f(x)}_{\geq}^{\prime \prime} &:= 2\theta_2 + 6 \theta_3 x + 6 \theta_4 (x-\alpha)\\
    {f(x)}_{\geq}^{\prime \prime \prime} &:= 6 \theta_3 + 6 \theta_4\\
\end{align*}
\begin{align*}
  {f(x)}_{<} &= {f(x)}_{\geq} \text{ } \checked \\
  {f(x)}_{<}^{\prime} &=   {f(x)}_{\geq}^{\prime} \text{ } \checked \\
  {f(x)}_{<}^{\prime \prime} &= {f(x)}_{\geq}^{\prime \prime}  \text{ } \checked \\
  {f(x)}_{<}^{\prime \prime \prime} &\neq {f(x)}_{\geq}^{\prime \prime \prime} \text{ } \times \\
\end{align*}
\end{document}
