%Vorlesung vom 21.11.17

\begin{document}
	\textbf{Wünsche: } \begin{itemize}
		\item stetige Übergänge an den Intervallgrenzen
		\item keine Knicke an den Intervallgrenzen
		\item [$\rightarrow$] glatter Übergang
	\end{itemize}

\subsection{Splines (polynome der Ordnung M)}
''Spline'' $\approxeq$ Polynomzug
\todo{Glätten Sprünge der Regressionsfunktionen s.o.}

\begin{itemize}
	\item $ \alpha_1 < \ldots < \alpha_r $
	\item[$\rightarrow$] $(r+1) \text{ Intervalle }  (-\infty, \alpha_1),[\alpha_1, \alpha_2) , \ldots, [\alpha_{r-1} , \alpha_r), [\alpha_r, \infty)$
	\item $Y_i  = \sum_{j=1}^{r} \theta_j h_j (x_i) + \epsilon_i$ , $i \in {1,\ldots,n}$
	\item [] 
	\begin{itemize}
		\item mit $h_1(x):=x^0$
		\item $h_2(x) := x^1$
		\item $h_{M+1}(x) := x^{M}$
		\item $h_{M+2}(x) := (x- \alpha_1)^{M}_{+}$
		\item $h_{M+r+1}(x) := (x- \alpha_r)^{M}_{+}$
	\end{itemize} 

\item[$\rightarrow$] Ableitungen bis zur Ordnung (M-1) in den Knoten stetig
\end{itemize}
\underline{Üblich:} 
\begin{itemize}
	\item Knoten wählt man äquidistant oder gemäß der Quantile
	\item $M \in \{0,1,3\}$
\end{itemize}

\todo{Grafik 1 einfügen }


\section{Multiples Testen}
\subsection{Rückblick}
\textbf{Hypothese:} \begin{itemize}
	\item [$\theta$] -Parameter von Interesse (z.B. $\theta = \mu$, $\theta = \sigma^2$, $ \theta = \beta_j $ , ....)
	\item[$\Theta$] Parameterraum für Parameter $\theta$ (z.B. $ \Theta = \mathbb{R}, \Theta = \mathbb{R}_+, \Theta= [0,1] $ ) 
\end{itemize}
Zerlegung vom $ \Theta $ in $ \Theta_0 $ und $ \Theta_A $, sodass gilt
\begin{itemize}
	\item $ \Theta_0 \cup \Theta_A = \Theta $
	\item $ \Theta_0 \cap \Theta_A = \emptyset $
	\item[$ \rightarrow $] $H_0$ $\theta \in \Theta_0 vs. H_a \theta \in \Theta_A $
\end{itemize}
(z.B $ \theta = \mu $,  $ \Theta = \mathbb{R} $, $ \Theta_0 := {0} $ , $ \Theta_A = \mathbb{R} \ {0} $)\\\\
\textbf{Test:} 
\begin{itemize}
	\item[] $ S^n $  -Raum aller n-elementigen Stichproben
	\item[] $ X = X_1,\ldots X_n)^T \in S^n $  -Stichprobe
	\item[] $ \phi : S^n \rightarrow {0,1} $ -Test, Abbildung von $ S^n $ - nach 0 bzw. 1
\item[$ \rightarrow $] Konvention: 
\begin{itemize}
	\item[]$ \phi(X) = 1 \Leftrightarrow H_0 $ wird verworfen
	\item[]$ \phi(X) = 0 \Leftrightarrow H_0 $ wird nicht verworfen
\end{itemize}
\end{itemize}
\textbf{Fehler:} 
\begin{itemize}
	\item Fehler 1.Art ($\alpha$ - Fehler, Typ 1 Fehler)
	\begin{itemize}
		\item $H_0$ ablehnen, obwohl $H_0$ gilt
		\item[$\rightarrow $] $\phi(X)=1$ aber $\theta \in \Theta_0$
	\end{itemize}	
	\item Fehler 2.Art ($\beta$-Fehler, Typ 2 Fehler)
	\begin{itemize}
		\item $H_0$ nicht ablehnen, obwohl $H_0$ nicht gilt
		\item[$\rightarrow$] $\phi(x) = 0$ aber $\theta \in \Theta_A$
	\end{itemize}
\end{itemize}
\textbf{Vorgehen:}
\begin{enumerate}
	\item Festlegen einer oberen Schranke $\alpha$ für den Fehler 1. Art (z.B. $\alpha$ = 5\% = 0.05)
	\item mit $\alpha$ auch 1. minimierung der Wahrscheinlichkeit $\beta$ für einen Fehler 2. Art
\end{enumerate}

\subsection{Multiples Testproblem}
\todo{Grafik multiples Testsystem einfügen}
\begin{exmp}
	Genetik, SWAS
	\begin{itemize}
		\item 500 000 SNPs werden auf Assoziation mit einem Phänotypen getestet
		\item $\alpha = 5\%$ $\rightarrow$ ca. 25 000 falsch positive Ergebnisse
	\end{itemize}
\end{exmp}
\newpage
Gegeben:
\begin{itemize}
	\item $X= {X_1, X_n}^T \in S^n$ - Stichprobe,
	\item $\theta$ -interessierende Parameter,
	\item $\Theta$ Parameterraum für $\theta$
\end{itemize}
$\mathcal{H}_0 := \{ H_0^{(j)} \subseteq \Theta$ , $j \in \{1, \ldots, m\} \}$ - Menge von Nullhypothesen\\
$\mathcal{H}_A := \{ H_A^(j) = \Theta \ H_0^{(j)} , j \in \{1, \ldots, m\}\}$ - Menge der entsprechenden Alternativhypothesen
\begin{itemize}
	\item[$\rightarrow$] Ein multipler Test ist ein Vektor 
	\item[]$\phi = (\phi_1, \phi_n)^T: S^n \rightarrowtail \{0,1\}^m$ 
wobei jedes $\phi_j$ mit $j \in \{1,m\}$ ein Test ist ($\phi_j: S^n \rightarrow \{0,1\}$) auf Grundlage der Stichprobe $X = (X_1,\ldots, X_1)^T$
\end{itemize}
\textbf{Interpretation}
\begin{itemize}
	\item Sei $\theta \in \Theta$ der wahre Parameter.
	\item[$\rightarrow$] $H_0^{j}$  gilt $\Leftrightarrow \theta \in H_0^{(j)} \subseteq \Theta$
	\item[$\rightarrow$] $H_0^{j}$ gilt nicht $\Leftrightarrow \theta \in H_A^{(j)} \subseteq \Theta \ H_0^{(j)} $
mit $j \in \{1,\ldots,m\}$
	\item[]$I_0(\theta) := \{j \in \{1,m\} : \theta \in H_0^{(j)} \}$ - Menge der unter $\theta$ wahren Nullhypothesen
	\item[]$I_A(\theta) := \{j \in \{1,m\} : \theta \in H_A^{(j)} \}$ - Menge der unter $\theta$ wahren Alternativhypothesen
\end{itemize}

\end{document}