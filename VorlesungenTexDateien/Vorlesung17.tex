
%Vorlesung vom 04.01.18
\section{Schätzen der Kovariablen mittels MLE (GLMs)}
Die log-likelihood für n unabhängige Beobachtungen $y_1,\cdots,y_n$ direkt aus der generalisierenden Form der Exponentioalverteilung erstellen:
\[ l(y|\theta, \phi) = \exp \left( \frac{y \theta - b(\theta)}{a(\phi)} + c(y,\phi) \right) \]
\[ log l (\theta, \phi, y) = \sum\limits_{i=1}^{n} log l (y_i | \theta, \phi) = \sum\limits_{i=1}^{n} \left( \frac{y \theta - b(\theta)}{a(\phi)} + c(y,\phi) \right) \]
\[theta_i = g(\mu_i)=\beta_0 + \beta_1 X_1,\dots,\beta_n X_n \]
$\Rightarrow$ Partielle Ableitungen der likelihood Funktion für einzelnen Regressionskoeffizienten berechnet
\section{Test auf signifikanten Effekt}
\begin{enumerate}
	\item Wald-Test
	\item Likelihood-Ratio-Test
\end{enumerate}

\subsection{Wald-Test}
Der Wald-Test ist eine generalisierte Form des t-tests. Hier ist der Wert der Teststatistik standardnormalverteilt $(\sim N(0,1))$.
\[Z_n = \frac{\bar{X} - \mu}{\sigma / \sqrt{n}} \]
Ein Maximum-likelihood-Schätzer kann als Mittelwert aus einer Summe einer großen Anzahl von Zufalsvariablen interpretiert werden.
Bei großen n die Summe von identisch verteilten und unabhängigen Zufallsvariablen asymptotisch normalverteilt ist (zentraler Grenzwertsatz).
\[\sum\limits_{i=1}^{n} X_i \sim N(n \cdot \mu, n \cdot \sigma^{2}) \]
Aus dem zentralen Grenzwertsatz folgt, dass die Mittelwerte von beliebig verteilten Zufallsvariablen standardnormalverteilt sind. \\
$\Rightarrow$ Kann für einen Ml-Schätzer der Wald-Test genommen werde, um den Wert des Schätzers mit dem Wert aus einer Grundgesamtheit abzugleichen

\[ H_0: \beta_i = 0 \]
\[ H_1: \beta_i \neq 0 \]
\[ w = \frac{\hat{\beta_{iML}-0}}{\sigma / \sqrt{n}} \sim N(0,1) \]
\[ w = \frac{\hat{\beta_{iML}^{2}-0}}{Var(\sigma / \sqrt{n})} \sim \chi_{df,1-\alpha}^{2} \]
\todo{ML im Index heißt der Koeffizient wurde mit Maximum Likelihood geschätzt. df im Index heißt degrees of freedom}
$\Rightarrow$ Da ML-Schätzer nur bei großen n standardnormalverteilt sind, ist der Wald-Test nur bei großen n zulässig
$\Rightarrow$ Wald-Test approximiert einen Likelihood-Ratio Test
$\Rightarrow$ Im Vergleich zum LR-Test muß nur ein Modell (mit allen Koeffizienten) gefittet werden
\section{LR-Test}
Vergleicht die likelihood des gesamten Modells mit der Likelihood des reduzierten Modells. Das reduzierte enthält den Koeffizienten, der getestet werden soll nicht.
\[ H_0 : \beta_i = 0 \Leftrightarrow g(\mu) = \beta_0 \]
\[ H_0 : \beta_i \neq 0 \Leftrightarrow g(\mu) = \beta_0 + \beta_1 X \]

\[LR = \frac{l(H_0 | x_1,\cdots,x_n}{l(H_1 | x_1,\cdots,x_n}\]
\[ -2 ln LR \sim  \chi_{df,1-\alpha}^{2} \]
$\Rightarrow$ LR-Test erfordert, dass beide Modelle $g(\mu)=\beta_0$ und $g(\mu)=\beta_0 + \beta_1 X$ gefittet werden müssen.
$\Rightarrow$ LR-Test st allerdings auch für geringe Stichprobengrößen geeignet. (Im Vergleich zum Wald-Test)

\section{Grundlagen der Versuchsplanung}
Alle gewonnen Beobachtungen für Y unterliegen einer Vielzahl von unterschiedlichen Einflußfaktoren. Einige davon sind bekannt (und können quantifiziert werden), andere dagegen sind unbekannt oder wirken zufällig auf die Beobachtungen $y_1,\cdots,y_n$
\todo[inline]{Stichprobenbild}
$\Rightarrow$ alle systematischen Effekte sollten bei der Versuchsplanung mit berücksichtigt werden 
Es müssen 3 Bedingungen erfüllt sein:
\begin{enumerate}
	\item Randomisierung von Proben/Individuen zwischen den Kategorien einer kategorialen Variable
	\item Es muß auf eine ausreichende Anzahl von Replikaten geachtet werden
	\item Falls systematische Effekte bei der Versuchsdurchführung auftreten können, müssen diese bei der Verteilung der Proben auf die Prozessierungsblöcke berücksichtgt werden (\underline{Blocking})
\end{enumerate}

