\documentclass[../VorlesungMaster.tex]{subfiles}
%Vorlesung 6
Der Zusammenhang nominaler Merkmale kann mit höherer Power mit dem (asymptotischen) $\chi^2$-Test als mit einem exakten Test überprüft werden.
\[ \chi^2 = \sum\limits_{i,j} \frac{(n_{ij} - e_{ij})^2}{e_{ij}} \]
\[ e_{ij}= \frac{n_{i\cdot} \times n_{\cdot j}}{n_{\cdot \cdot}} \]
$e_{ij}:$ erwartete Anzahl \\
$\chi^2$ hat $\chi^2$-Verteilung mit $f=(l-1)\cdot(k-1)$ Freiheitsgrade \todo{l= Anzahl an Zeilen, k= Anzahl Spalten der Kontigenztafel}\\
Faustregel: $n_{ij} \neq 0$ für alle $i,j$ und $< 20 \%$ der Zellen haben $l_{ij} < 5 \rightarrow \chi^2$-Test \\
%Ende der Vorlesung
