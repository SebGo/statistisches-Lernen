\chapter{Deskriptive Statistik}
\begin{itemize}
	\item Beschreibung von Daten und Kohorten ist zentral für das Verständnis einer Arbeit
	\item Ziel ist mit wenigen Kenngrößen das wesentliche zu charakterisieren 
	\item man gibt ''Punktschätzer''  für Erwartungswerte und ''Konfidenzintervalle'' (KI; engl. CI) als Maß für die Genauigkeit der Schätzung:
	\[(1- \alpha) \text{ KI} [a,b] : P(a \leq \theta \leq b) = 1-\alpha\]
\end{itemize}
\section{Nominale und Ordinale Größen}
%nominal -- Kann man nur kategorisieren aber nicht ordnen
%ordinal -- ist eine Kategoriale Größe die man in eine Reihenfolge bringen kann

\begin{itemize}
	\item absolute und relative Häufigkeiten (z.B Häufigkeitstabellen)
	\item grafisch: Balkendiagramme (mit Konfidenzintervall KI oder Standardfehler SE)
	\item Kreisdiagramm (verpönt)
\end{itemize}
\begin{align*}	
	\hat{p} &= \frac{r}{n}\\
	\hat{SE} &= \frac{\hat{sd}}{\sqrt{n}}= \frac{\sqrt{\hat{V}}}{\sqrt{n}} = \sqrt{\frac{\hat{p} (1- \hat{p})}{n}}
	KI \approx \hat{p} \pm 2 SE,
\end{align*}
r= \#Fehlgeburten, n = \#Beobachtungen 



\section{Metrische Daten}
\textbf{Lagemaß}
\begin{itemize}
	\item Mittelwert (arithmetisch oder geometrisch, d.h. log-Skala)
	\begin{itemize}
		\item übliche und ''robuste'' Methoden
		\item arithmetisch  = $\frac{1}{n}\sum_{i=1}^{n} X_i$
		\item geometrisch  = $[ \prod_{i=1}^{n} X_i]^{1/n}$
	\end{itemize}
	-Median und andere Quantile (verschiedene Schätzverfahren)
\end{itemize}
\textbf{Streumaß}
\begin{itemize}
	\item Standardabweichung (''sample'' - Methode) $sd^2 = \frac{1}{n-1} \sum_{i=1}^{n}(x_i - \bar{x})^2$
	\item Interquartilabstand (engl. interquartile range (IQR) z.B. 25. und 75. Perzentil)
	\item Spannweite
	\item grafisch: Histogramm, Boxplot	
\end{itemize}

\section{Zusammenhang 2er Merkmale}
Nominal
\begin{itemize}
	\item Kontigenztafel: odds ratio, relatives Risiko
	\item grafisch: forest plots
\end{itemize}
metrisch
\begin{itemize}
	\item Korrelationskoeffizient (mit KI)
	\item Streudiagramm
\end{itemize}

\paragraph{Simpsons Paradoxon}
Grundidee: Effekt in Gesamtgruppe muss nicht ''echt'' sein. Er kann in Subgruppen anders ausfallen.

\begin{tabular}{|c|c|c|}
	\hline 
	& A & B \\ 
	\hline 
	Erfolg & 70 (30\%) &  50(22\%) \\ 
	\hline 
	Misserfolg & 160 & 182 \\ 
	\hline 
	Summe & 230 & 232 \\ 
	\hline 
\end{tabular} 

\begin{tabular}{|c|c|c|c|}
	\hline 
	&   & A & B \\ 
	\hline 
	Männer & E & 7 (20\%) & 45(21\%) \\ 
	\hline 
	& M & 28 & 175 \\ 
	\hline 
	Frauen & E & 63(32\%) & 5(33\%) \\ 
	\hline 
	& M  & 132 & 10 \\ 
	\hline 
\end{tabular} 

