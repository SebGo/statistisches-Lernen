%Vorlesung vom 30.11.17

\begin{document}
\section{Wdh.: Multiples Testen}
\underline{Gegeben:} 
\begin{itemize}
 \item $X=(X_1,...,X_n)^T \in S^n$
 \item $H_0$ - Nullhypothese
 \item $H_A$ - Alternativhypothese
 \item 1 Test \\ $\rightarrow$ Fehler 1.Art: $P(\phi=1,\theta \in H_0) \leq \alpha$
 \item $m > 1$ Tests $\rightarrow$ FWER=$P \left(\bigcup_{j \in I_0(\theta)} \{ \phi_j = 1 \} \right) > \alpha$
 \begin{itemize}
  \item Bonferroni
  \item \v Sid\'ak
  \item Bonferroni-Holm(``step-down''-Verfahren)
 \end{itemize}
 $\rightarrow \text{ FDR}_{\theta}(\phi) := E \frac{V(\theta)}{\max\{R(\theta),1\}}$
\end{itemize}
[Nebenbemerkung: Falls für alle $j \in  \{1,...,m\} \theta \in H_0^{(j)}, \text{dann gilt FWER}_{\theta}(\phi)=\text{FDR}_{\theta}(\phi)$ ]
\paragraph{Interpretation:}
Für $\alpha=0,05$ gilt dann $\phi$ liefert unter 100 Verwerfungen im Mittel maximal 5 fälschlicherweise verworfene Nullhypothesen
\section{Benjamin-Hochberg-Test}
\begin{itemize}
 \item $\phi=(\phi_1,...,\phi_m)^{T}$ - multipler Test 
 \item $p=(p_1,...,p_m)^{T}$ - zu Test gehörende p-Werte 
 \item $p_{(q)} \leq ... \leq p_{(m)}$ - geordnete p-Werte 
 \item $\tilde{\alpha_{j}} := j \frac{a}{m} , j \in \{1,...,m\}$ 
 \item $\phi^{FDR} = \left(\phi_{1}^{FDR},...,\phi_{m}^{FDR}\right)^{T}$ mit 
 \begin{itemize}
  \item $\phi_{1}^{FDR} = \begin{cases}
                           1, &  \text{ falls } j < j^* \\
                           0, &  \text{ falls } j \geq j^*
                          \end{cases}$ mit
 \item $j^{*}:=\min \{ \min \{j \in \{1,...,m \}: p_{(k)} > \tilde{\alpha}_{k}$ für alle $k \in \{j,...,m\} \}, m+1 \}$
 \end{itemize}
\item $P_j(X)$ unabhängig, $j \in \{1,...,m\}$
\end{itemize}

\begin{exmp}
\[ p_{(1)} \leq \tilde{\alpha}_1  = 1 \cdot \frac{\alpha}{6} \textcolor{green}{H_0 \text{ ablehnen}} \]
\[ p_{(2)} > \tilde{\alpha}_2  = 2 \cdot \frac{\alpha}{6} \textcolor{green}{H_0 \text{ ablehnen}}\]
\[ p_{(3)} \leq \tilde{\alpha}_3  = 3 \cdot \frac{\alpha}{6} \textcolor{green}{H_0 \text{ ablehnen}}\]
\[\textcolor{red}{j^*}\hspace{0.5cm} p_{(4)} > \tilde{\alpha}_4  = 4 \cdot \frac{\alpha}{6} \textcolor{green}{H_0 \text{ NICHT ablehnen}}\]
\[ p_{(5)} > \tilde{\alpha}_5  = 5 \cdot \frac{\alpha}{6} \textcolor{green}{H_0 \text{ NICHT ablehnen}}\]
\[ p_{(6)} > \tilde{\alpha}_6  = 6 \cdot \frac{\alpha}{6} \textcolor{green}{H_0 \text{ NICHT ablehnen}}\]
\end{exmp}

$\Rightarrow \phi^{FDR}$ ist FDR-kontrollierend zum Niveau $\alpha$  \\
$\Rightarrow$ ``Step-up''-Verfahren

\section{Modellwahl und Regularisierung}
\underline{Hier:} Multiple lineare Regression 
\[ Y_{i} = \beta_{0} + \beta_{1} x_{i}^{(1)} + ... \beta_{k} x_{i}^{(k)} + \epsilon_{i} i \in \{ 1,...,n \} \]
Schätzer für $\beta_{j}$ über kleinste-Quadrate-Methode\\
\underline{Problem:}
\begin{itemize}
 \item $n \approx k$
 \item $k > n$
\end{itemize}
$\rightarrow$ 
\begin{itemize}
 \item $k > n$: keine eindeutige Lösung
 \item hohe Varianz bei den Schätzern $\widehat{\beta_j}$
 \item schlechte Qualität bei Vorhersagen
\end{itemize}

\underline{Lösung:}
\begin{itemize}
 \item eliminieren irrelevanter Variablen
 \item zusätzliche Bedingungen in das Modell einfügen
\end{itemize}

\subsection{Modellwahl}
\subsubsection{Modellbewertung}
\paragraph{Residual Sum of Squares(RSS)}
\[ \text{RSS}:= \sum\limits_{i=1}^{n} (y_i \hat{y_i})^2 \text{ mit} \]
\[ \hat{y}_i =  \hat{\beta_{0}} + \hat{\beta_{1}} x_{i}^{(1)} + ... \hat{\beta_{k}} x_{i}^{(k)} + \epsilon_{i} i \in \{ 1,...,n \} \]
$\Rightarrow$ ``je kleiner desto besser''
\end{document}
