\documentclass[10pt,a4paper]{article}
\usepackage[latin1]{inputenc}
\usepackage{amsmath}
\usepackage{amsfonts}
\usepackage{amssymb}
\usepackage{graphicx}
\usepackage{geometry}
\usepackage{tikz}
\usepackage{todonotes}
\usepackage{amsthm}

\geometry{a4paper, top=25mm, left=40mm, right=25mm, bottom=30mm,
	headsep=10mm, footskip=12mm}
\title{Uebung 2 - Statistisches Lernen\\ \large{Wintersemester 17/18}}
\date{}

\begin{document}
	\maketitle
\section{Aufgabe 1}
\[P(D^+ | T^+) = \frac{P(T+|D+) P(D^+) }{ P(T^+|D^+) P(D^+) + P(T^+ | D^-) P(D^-) } = \frac{Se \cdot Pr}{ Se \cdot Pr + (1-Sp) \cdot (1-Pr)} \]
Für $Pr=0.01, Se=0.98, Sp=0.98 \rightarrow  P(D^+ | T^+)= \frac{1}{3} $
\[P(D^+ | T^+ T^+)=\frac{P(T^+ T^+ | D^+) P(D^+)}{P(T^+ T^+ | D^+) P(D^+) + P(T^+T^+|D^-) P(D^-)} \]
\[P(T^+ T^+| D^+) = \left( P(T^+|D^+) \right)^2 + \rho P(T^+ | D^+)(1-P(T^+|D^+)) = (1 - \rho)(P(T^+|D^+))^2 + \rho P(T^+|D^+) \]
\[ P(T^+ T^+|D^-) = (1- \rho) \cdot (P(T^+ | D^-)^2 + \rho \cdot P(T^+ | D^-) \]
\[ = (1 - \rho) (1-P(T^-|D^-))^2 + \rho (1-P(T^-|D^-)) \]
\[ = 1 + (\rho -2) P(T^- | D^-) + (1- \rho) (P(T^-|D^-))^2\]
\[ P(D^+ | T^+ T^+)= \frac{((1-\rho) \cdot Se^2 + \rho \cdot Se) \cdot Pr}{((1-\rho)Se^2 + \rho \cdot Se)\cdot Pr + ((1-\rho)(1-Sp)^2 + \rho(1-sp))\cdot (1-Pr)} \]

\[ \sigma = 0: P(D^+|T^+T^+)= \frac{Se^2 \cdot Pr}{Se^2 \cdot Pr + (1-Sp)^2(1-Pr)} \sim 0.96 \]
\[ \sigma = 1: P(D^+|T^+T^+)= \frac{Se \cdot Pr}{Se \cdot Pr + (1-Sp)(1-Pr)} \sim \frac{1}{3} \]
\[ \sigma = 0.7: P(D^+|T^+T^+)= ... = 0.4107 \]

\section{Aufgabe 2}
\begin{tabular}{c|cccccccccc|c}
A & 0 & 0 & 0 & 114 & \multicolumn{6}{c|}{$\vdash $ MW = 6.64 $ \dashv$} &170 \\ \hline
B & 0 & 0 & 0 & 108 & \multicolumn{6}{c|}{$\vdash $ MW = 7.88 $ \dashv$} &170 \\ \hline
Anzahl FG & 0 & 1 & 2 & 3 & 4 & 5 & 6 & 7 & 8 & 9..... &  \\ 
\end{tabular} 

Woher kennt man diese Mittelwerte?
\[ \frac{1}{170} \sum\limits_{i=1}^{170} n_i^A \]
\[ = \frac{1}{170} \left( \sum\limits_{i=1}^{114} 3 + \sum\limits_{i=115}^{170} n_i^A \right) \]
\[ \Rightarrow \frac{1}{170} \sum\limits_{i=115}^{170} n_i^A = \frac{1}{56} \left[ 170 \cdot 4.2 - 3 \cdot 114 \right] \]

\section{Aufgabe 3}
Anzahl der Diabetiker steigt mit wachsendem Steatosegrad \\
Wenn man die Diabetiker nach Krankheiten aufschlüsselt verschwindet dieser Trend

\end{document}